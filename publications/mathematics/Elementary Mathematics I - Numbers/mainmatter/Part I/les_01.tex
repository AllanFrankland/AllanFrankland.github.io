\section{Integers: An Intuitive Definition}

\begin{introduction}
    We begin with an instuitive definition of the integers. We start by introducing the positive numbers as
    those numbers most familiar to us and move progressively into the abstract integers.
\end{introduction}

The most familiar numbers are those used in everyday life

  $$
    1,2,3,4,\dots
  $$

called the \textbf{positive numbers}. Another familiar number is

  $$
    0.
  $$

Taking together the positives numbers and zero we have what are called the \textbf{natural numbers}

  $$
    0,1,2,3,\dots
  $$

These numbers represent only symbols that mean nothing to us until we give them meaning.
We do this by relating each number to the other on a number-line:

\begin{figure}[h]
  \centering
    \begin{tikzpicture}
      \draw[-] (0,0) -- (5.,0) node[right] {};
      \foreach \x in {1,2,3,4,}
      \draw (\x,0.1) -- (\x,-0.1) node[below] {\x};
      \draw (0,0.1) -- (0,-0.1) node[below] {0};
    \end{tikzpicture}
    \caption{Natural Numbers}
    \label{fig: natural numbers}
\end{figure}

Each number is placed at equal intervals apart along the line; the distance from $0 \to 1$ is the same as the
distance from $1 \to 2$ and so forth. We call zero the \textbf{origin} and each number represents its distance
from the origin; $4$ is that number which is four intervals away from the origin.

If we extend the number-line to the left we get the \textbf{negative integers}\footnote{It is worth noting here
that 'negative' and 'minus' are often used interchangeably but, following Serge Lange's distate for this mistake,
I will not be using them to mean the same thing. See Q4.}

\begin{figure}[h]
  \centering
    \begin{tikzpicture}
      \draw[-] (-4,0) -- (4.,0) node[right] {};
      \foreach \x in {-3,-2,-1,1,2,3,}
      \draw (\x,0.1) -- (\x,-0.1) node[below] {\x};
      \draw (0,0.1) -- (0,-0.1) node[below] {0};
    \end{tikzpicture}
    \caption{Integers}
    \label{fig: Integers}
\end{figure}

and when we take all of these togethers (positive integers, natural numbers, negative integers) we get the
set\footnote{A technical term representing a concept we are not yet prepared to deal with, but for now
can be taken to mean 'collection', 'class' or 'species' } of numbers called the \textbf{integers}.

\newpage

\subsection{Addition and Algebra Introduction}

\begin{introduction}
  Now we move onto performing operations on the integers which enable us to both manipulate the integers and
  discovers its rules.
\end{introduction}

Addition is the most fundamental operation we can perform on the integers which involves moving along the
number-line (in any direction) according to a certain amount, we denote the operation with the $+$ symbol.

Let us begin by adding zero to any number:

$$
  5 + 0 = 5.
$$

We don't move left or right and remain precisely where we started. This leads to our first axiom of the integers.

\begin{axiom}{Additive Identity of the Integers}

  Zero added to any number results in that same number.

\end{axiom}

Notice here that we stated this rule using words\footnote{This was actually common before the creation of
algebraic notation.\textit{The Compendious Book on Calculation by Completion and Balancing} is a 9\textsuperscript{th}
century algebra text written entirely in prose and is the source of our word 'algebra'.} which makes our rule
excessively verbose and muddy. It would be convenient to introduce a notation for speaking about any numbers of a set.
Let us do that by restating our axiom in this new notation

\setcounter{axiom}{0}
\begin{axiom}{Additive Identity of the Integers}

  For any integer $a$, we have
  $$
    0 + a = a.
  $$
\end{axiom}

We introduce the letter $a$ to represent any integer; we do not care if its $0$, $1$, $-1442$ or any integer
you can think of, so long as \textit{it is is an integer} you can replace $a$ with anything and the axiom applies. We
will use this notation going forward.

\subsection{Rules ofAdding Integers}

\section{Socratic Questions}

\begin{qa}
  \question{What are these symbols we call numbers?}
  \answer{

  The symbols themselves $(0, 1, 2,)$ etc are a conventional notation used to represent a understood distance
  away from the origin. This notation is Arabic, but other notation systems exist such as Roman Numerals , Egyptian,
  etc. Our numeric system is base-$10$ and \textit{positional} which means we use ten digits $(0$ through $9)$ and
  the position of a digit in a number represents its \textit{place-value}.

  }

  \question{What does $\dots$ mean?}
  \answer{
    This is an ellipsis that is shorthand notation for "and so on" or "so forth". It indicates that
    continuation of a sequence beyond what is written explicitly. Likewise the line extending past the
    final number on the number-line plays a similar role

    }

  \question{What is an axiom?}
  \answer{
    Axiom is mathematical jargon for a rule that is asserted to be true but not proven. It is an assumption we
    make from which we then extrapolate new rules. The axioms governing integers for everyday mathematicians are
    the Peano Axioms, by Giuseppe Peano. Axioms are not empirical claims, but premises and, as such, cannot be
    disproven in the oridnary sense but can be rejected. A familiar axiom to you that has since been
    rejected is Euclids axiom that two parallel lines never meet - this is no longer accepted wholesale, such as
    in the field of non-Euclidean geometry.
  }

  \question{Why 'minus' and not 'negative'}
  \answer{
    I use 'minus' for $-a$ wherever the sign\footnote{'sign' refers to the positivity or negativity of a number} of
    the value represented by $a$ is ambiguous. Given an integer $a$ the phrase 'negative $a$' carries with it the
    implication that the value represented by $a$ is negative, that is it exists to the left of zero on the
    number-line. The phrase 'minus $a$' carries with it no such implication and instead refers to the operation.

    This may sound like a nit-pick but clear nomenclature makes the busy work of mathematics less cumbersome
    later on when performing large operations and multi-step deductions where such simple mistakes are common.
  }
\end{qa}




































%\section{Mathematics}
%This template utilises a whole assortment of custom \index{mathematics} commands and
%enviroments for page structure purposes. The main \index{mathematics!commands} commands
%can be found in \index{mathematics!newcommands.sty} newcommans.sty and
%\index{mathematics!Environemnts} theorem environments in preamable.sty.
%Here is a list of all environments:
%\begin{proof}
%  This is a \index{mathematics!Environments!proof} proof environment.
%\end{proof}
%\begin{axiom}[descriptor]
%  This is an \index{mathematics!Environments!!axiom} axiom environment. Numbered using
%  Roman numerals.
%\end{axiom}
%\begin{cor}[descriptor]
%  This is an \index{mathematics!Environments!corrolary} corollary environment.
%\end{cor}
%\begin{conjecture}[descriptor]
%  This is an \index{mathematics!Environments!conjecture} conjecture environment.
%\end{conjecture}
%\begin{define}[descriptor]
%  This is an \index{mathematics!Environments!define} define environment.
%\end{define}
%\begin{ndefine}
%  This is an unumbered definition environment.
%\end{ndefine}
%\begin{eg}
%  This is an \index{mathematics!Environments!eg} example environment.
%\end{eg}
%\begin{solution}
%  This is an \index{mathematics!Environments!solution} solution environment.
%\end{solution}
%\begin{lemma}
%  This is an \index{mathematics!Environments!lemma} lemma environment.
%\end{lemma}
%\begin{notation}
%  This is an \index{mathematics!Environments!notation} notation environment.
%\end{notation}
%\begin{thm}
%  This is an \index{mathematics!Environments!thm} theorem environment.
%\end{thm}
%
%\subsection{Tables}
%This template utilises extensive use of table packages for custom and better looking
%\index{tables} tables.
%\begin{table}[ht]
%  \centering
%  \begin{tabular}{lll}
%      \toprule
%      \multicolumn{1}{c@{}}{Law} & \multicolumn{2}{c@{}}{Forms}                                                                                \\
%      \addlinespace[0pt]
%      \cmidrule(lr){1-1}\cmidrule{2-3}
%      Commutative                & $p\wedge q\equiv q\wedge p$                           & $p\vee q\equiv q\vee p$                             \\
%      Associative                & $(p\wedge q)\wedge r\equiv p\wedge (q\wedge r)$       & $(p\vee q)\vee r\equiv p\vee (q\vee r)$             \\
%      Distributive               & $p\wedge (q\vee r)\equiv (p\wedge q)\vee (p\wedge r)$ & $p\vee (q\wedge r)\equiv (p\vee q)\wedge (p\vee r)$ \\
%      Identity                   & $P\wedge t\equiv p$                                   & $p\vee c\equiv p$                                   \\
%      Negation                   & $p\vee\neg p\equiv t$                                 & $p\wedge\neg p = c$                                 \\
%      Double negative            & $\neg(\neg p)\equiv p$                                &                                                     \\
%      Indempotent                & $p\wedge p\equiv p$                                   & $p\vee p\equiv p$                                   \\
%      Universal bound            & $p\vee t\equiv t$                                     & $p\wedge c \equiv c$                                \\
%      De Morgan's                & $\neg(p\wedge q)\equiv \neg p\vee \neg q$             & $\neg(p\vee q)\equiv \neg p \wedge \neg q$          \\
%      Absorption                 & $p\vee(p\wedge q)\equiv p$                            & $p\wedge(p\vee q)\equiv p$                          \\
%      Negation of $t$ and $c$    & $\neg t=c$                                            & $\neg c = t$                                        \\
%      \bottomrule
%  \end{tabular}
%  \caption{Laws of Logic}
%  \label{table: Laws of Logic}
%\end{table}
%
%\section{Referencing}
%Utilising the varioref package this template can have intelligent \index{cross references}
%cross references. The command vref can be used for intelligent references that will
%automatically change within the context of the template. For example: See
%\vref{table: Laws of Logic} for laws of logic.
%
%Likewise I also use \index{Bibtex} Bibtex for reference buildine, for example: Einstein was
%cool\cite{OSullivan2009}. Note that all cross-references, references and table of content
%sections are clickable and allow you to jump to these sections and or references.
%
%The (numbered) \verb|Theorem| and \verb|Definition| environments support automatic
%labelling. For example definition \vref{def: 2.1} is labelled indirectly via
%the definition of its environment. We can also reference theorems like the theorem
%\vref{thm: 2.1}.
%
%\index{cross references!Equations} Equations are listed by section, for example:
%\begin{equation}\label{eq: 3.1}
%  a + b = b + a.
%\end{equation}
%
%The equation \vref{eq: 3.1}\footnote{Notice that the vref is clickable and will send you to
%the equation it references} is the definition of the commutative property.
%
%\section{Code}
%This template supports a custon \index{code} code environenment that supports c++ highlighting
%by default:
%\begin{lstlisting}[]
%  #include <cstdlib>
%  #include <ctime>
%  #include <iostream>
%  using namespace std;
%  double doubleRand( ) {
%    return double(rand( )) / (double(RAND_MAX) + 1.0);
%  }
%  int main( ) {
%    srand(static_cast<unsigned int>(clock( )));
%    cout << "expect 5 numbers within the interval [0.0, 1.0)" << endl;
%    for (int i=0; i < 5; i++) {
%      cout << doubleRand( ) << "\n";
%    }
%  cout << endl;
%}
%\end{lstlisting}
%
%\section{Indexing}
%This template supports \index{indexing} index. The package \verb|imakeidx| is used to build
%the index - \verb|makeindex| in the preamable - and we use \verb|\printindex| located within
%\verb|backmatter/index.tex| to print the index wherever in our document we wish to print it.
%Note that use of \verb|\index{key}| will mean that \verb|key| does not show up in the
%final version; you effectively need to write \verb|\index{key} key| to have the \verb|key|
%appear in text. s
%
%The \verb|imakeidx| package has some key useful \index{indexing!commands} commands:
%
%\begin{itemize}
%  \item \verb|\index{key}| - Adds the \verb|key| to the index, the \verb|key| is any line
%        of text
%  \item \verb|\index{pkey!ckey} - Adds \verb|ckey| to the index as a child of \verb|pkey|,
%        where \verb|pkey| has already been added to the index.
%  \item \verb~\index{key|textbf}~ - Adds the current page number to \verb|key| in the index,
%        allows for indexing a single \verb|key| on multiple pages.
%\end{itemize}
%
%The template also supports multiple indexes, for exaple you can have a seperate index
%for defintiions:\index[definition]{Commutativity}
%\begin{define}[Commutativity]
%  \[
%    a + b = b + a
%  \]
%\end{define}